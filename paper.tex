\documentclass{article}


% if you need to pass options to natbib, use, e.g.:
%     \PassOptionsToPackage{numbers, compress}{natbib}
% before loading neurips_2024

% submissions should not be anonymous, so use the 
% [preprint] option:
\usepackage[preprint]{paper}


% to avoid loading the natbib package, add option nonatbib:
%    \usepackage[nonatbib,preprint]{neurips_2024}


\usepackage[utf8]{inputenc} % allow utf-8 input
\usepackage[T1]{fontenc}    % use 8-bit T1 fonts
\usepackage{hyperref}       % hyperlinks
\usepackage{url}            % simple URL typesetting
\usepackage{booktabs}       % professional-quality tables
\usepackage{amsfonts}       % blackboard math symbols
\usepackage{nicefrac}       % compact symbols for 1/2, etc.
\usepackage{microtype}      % microtypography
\usepackage{xcolor}         % colors


\title{Smart Model Elimination for Efficient Automated Machine Learning}


% The \author macro works with any number of authors. There are two commands
% used to separate the names and addresses of multiple authors: \And and \AND.
%
% Using \And between authors leaves it to LaTeX to determine where to break the
% lines. Using \AND forces a line break at that point. So, if LaTeX puts 3 of 4
% authors names on the first line, and the last on the second line, try using
% \AND instead of \And before the third author name.


\author{%
  Eric Su Zhang\thanks{Use footnote for providing further information
    about author (webpage, alternative address)---\emph{not} for acknowledging
    funding agencies.} \\
  St. Mark's School of Texas\\
  10600 Preston Rd, Dallas, TX 75230\\
  \texttt{ericspring08@gmail.com} \\
  % examples of more authors
  \And
  Benjamin Standefer \\
  St. Mark's School of Texas \\
  10600 Preston Rd, Dallas, TX 75230 \\
  \texttt{bjmstandefer@gmail.com} \\
  % \AND
  % Coauthor \\
  % Affiliation \\
  % Address \\
  % \texttt{email} \\
  % \And
  % Coauthor \\
  % Affiliation \\
  % Address \\
  % \texttt{email} \\
  % \And
  % Coauthor \\
  % Affiliation \\
  % Address \\
  % \texttt{email} \\
}


\begin{document}


\maketitle


\begin{abstract}
  Automated Machine Learning or AutoML has emerged as a popular field of research. Many of these frameworks optimize a form of model selection, however, they all require every model to be run and evaluated. We propose a novel framework that automatically eliminates models that are unlikely to be performant by training a Boosting Model on hundreds of kaggle datasets. Out of the 30 model options, our framework can predict a top five model within its top eight 80\% of the time.
\end{abstract}


\section{Introduction}

Machine learning (ML) is a type of artificial intelligence (AI) that uses a variety of algorithms to interpret data and extrapolate from it, making predictions based on observed trends. Individual models are trained on datasets to become smarter so as to make predictions based on new data. The development and optimization of these models is a time-consuming process involving statistical nuances and complex learning strategies. This has led to the emergence of Automated Machine Learning (AutoML) frameworks and research. 

One sector of the global industry that has the most potential for beneficial integration of AutoML frameworks is healthcare, particularly in developing countries. These countries have severe shortages of healthcare workers and limited tools for diagnosis. For example, Africa has 2.3 healthcare workers per 1000 individuals, while the Americas have 24.8 healthcare workers per 1000. The World Health Organization (WHO) emphasized that this deficit is growing every year and will likely reach 18 million personnel by 2030. The use of AI in healthcare has been varied. Medical AI's typically automate repetitive tasks, making time consumption a primary concern. This coupled with a lack of adequate resources makes designing faster diagnostic systems for developing countries' medical sectors a pivotal issue. 

Most AutoML frameworks implement some form of model selection where a pool of models are filtered until a final model is selected. However, these frameworks require that every model to be run to filter out. In theory, this means that much energy is spent training models that are unlikely to be selected. We propose SMEML, a novel model selection algorithm that automatically eliminates models that it believes will not be performant. 

\subsection{Survey}

\documentclass{article}
\usepackage{geometry}
\usepackage{longtable}
\usepackage{graphicx}

\geometry{a4paper, margin=1in}

\begin{document}

\begin{longtable}{|l|l|l|l|l|}
\hline
\textbf{Factor} & \textbf{H2O} & \textbf{TPOT} & \textbf{MLJAR} & \textbf{FLAML} & \textbf{LightAutoML} & \textbf{GAMA} \\
\hline
\endfirsthead

\multicolumn{5}{c}%
{\tablename\ \thetable\ -- \textit{Continued from previous page}} \\
\hline
\textbf{Factor} & \textbf{H2O AutoML} & \textbf{TPOT} & \textbf{MLJAR} & \textbf{FLAML} & \textbf{LightAutoML} & \textbf{GAMA} \\
\hline
\endhead

\hline \multicolumn{5}{|r|}{\textit{Continued on next page}} \\
\hline
\endfoot

\hline
\endlastfoot

\textbf{Ease of Use} & Easy-to-use GUI and API & Python library, requires coding experience & Very user-friendly web interface, detailed API & Simple API & User-friendly API with minimal coding & Python library with fairly intuitive API\\
\hline
\textbf{Supported Algorithms} & Linear, Tree, DNN, Clustering, etc. & Genetic Programming & Linear, Tree, LightGBM, CatBoost, Clustering, etc. (no DNN) & Linear, Tree, Clustering - More limited and lightweight & Linear, LightGBM, Catboost, etc. & Linear, tree, SVM, etc. \\
\hline
\textbf{Feature Engineering} & Basic & Basic & Advanced - includes feature interactions & Limited & Advanced (automatic) & Basic \\
\hline
\textbf{Time Management} & Time Constraints are Settable & Generations and population size & Time constraints are Settable & Optimized for quick results & Settable time constraints & Settable time constraints\\
\textbf{Ensemble Methods} & Yes (stacking + blending) & Yes (stacking) & Yes (stacking + blending) & No & Yes (stacking) & Yes (stacking) \\
\hline
\textbf{Hyperparameter Tuning} & Automated & Genetic algorithm & Automated and customizable & Automated and efficient tuning & Automated & Automated \\
\hline
\textbf{Model Interpretability} & SHAP, LIME, PDPs, MOJO & POJO, etc. & Basic & SHAP, LIME, PDPs, etc. & Basic (permutation importance) & SHAP, PDPs, etc. & Basic (feature importance) \\
\hline
\hline
\textbf{Scalability} & Good (distributed computing) & Limited & Good & Good (resource-efficient) & Good (distributing computing) & Moderate \\
\hline
\textbf{Customization} & High & Moderate & High & Low-Moderate & Moderate & Moderate \\
\hline

\end{longtable}

\end{document}



\section{Methodology}

\subsection{Data Collection}

\subsection{Model Elimination}


\section{Experiments}

\section{Discussion}

\section{Conclusion}

\begin{ack}

\end{ack}

\section*{References}

\medskip
{
\small

% references here

\end{document}
